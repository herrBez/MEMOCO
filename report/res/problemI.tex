\section{First Assignment}
The purpose of this assignment was to implement an exact algorithm to solve the problem using the CPLEX API.
In order to call the solve function of the CPLEX library, previously I have to set up the environment
and the problem, i.e. pass the data to CPLEX. This is done by means of the \verb|setupLP| function.
Since we are interested (only) in the objective function and $y_{ij}$ variables, i.e. if the arc from $i$
to $j$ is used, the function takes
an additional\footnote{w.r.t. to the given skeleton} parameter,  \verb|mapY|. It is used to
retrieve from the CPLEX API the values of those variables, if the option  \verb|--print| is enabled.
The set up phase consists in generating the variables and the constraints and gives them to CPLEX.

\paragraph{ Variables and constraints}
The variables of the problem are $x_{ij}$ and $y_{ij}$ $\forall i,j \in N$.
In order to try to render the solution more efficient the program does not pass
to CPLEX the variables with index $i,i$. Indeed these arcs can be ignored because
they will never be used.



\paragraph{Note about Data-Structure} In order to render the reference to generated variables easier
during the constraint generation phase, I used two bi-dimensional vector \verb|mapX|, \verb|mapY| that contains in position $ij$ 
the corresponding CPLEX-variable index. They are filled during the variable generation phase.

The maps are perfect squares ($N\times N$), but the entries in position $i,i$ are not passed to CPLEX.
The advantage of using this approach is that $2*N$ variables are spared ($N$ for $x$ and $N$ for $y$) and the constraints
contain less variables.
The drawback is evident, when one write a function that reads the variables back from CPLEX with this configuration. He easily realizes
that it is challenging and hard to read.
You can check it out in the \verb|fetch_and_print_y_variables| function.
Since this functionality was not requested explicitly by the assignment,
this drawback can be ignored. Thus this approach is worth the effort.



