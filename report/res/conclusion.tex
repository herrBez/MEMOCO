\section{Conclusion}
Inside this report it was shown that CPLEX can work
very efficiently with small instances and random instances
in general. 
During the measures it turns out evident the CPLEX perfoms
in general worse and in an unpredictable way, when the holes are cluestered in the corners 
of the board. 

The tuning and analysis of the efficiency and stability of the population based solution was extremely
simplified by fixing the majority of the parameters. 
The genetic algorithm is very fast and provides a solution in brief time even if the input is big, especially
when the population size is set to 10. 
With the increasing of the problem size the precision of the genetic algorithm becomes worse and worse.

In general I can state  that with a little problem size it does not make sense to write and try to optimize
a genetic algorithm.

I assume that a company that produces PBCs will replicate the given board millions of time. Considering that, even if it will take a lot of time to the producer to get the exact solution the first time, at the end they will be much faster. The initial investment will pay off rapidly.

In general this kind of companies do not produce boards with several hundreds of holes, but more likely with less than 100\footnote{At least in the real world example.}.
The user time reported refers to a five years old laptop, but a company can surely rely on a better hardware.


 




