\section{Second Assignment}
In this second assignment the purpose was to implement a solution for the problem
using a metaheuristic approach.
I chose to use a genetic algorithm TODO aggiungi motivazoione
\paragraph{Encoding of the individuals.} In order to encode the individuals I used the provided
encoding, i.e. an array of dimension N+1 containing a sequence of holes index, e.g.
$<0\ 1\ \dots\ N\ 0>$
\paragraph{Initial Population.} The initial population is formed by $k$ individuals.
In order to diversify the population, fast all the individuals are chosen randomly, but
the increasing sequence $<0\ 1\ \dots\ N\ 0>$ and the decreasing sequence $<0\ N\ \dots\ N\ 0>$
are always granted. In order to start with better solution some of them are trained.

\paragraph{Fitness Function.} For simplicity the fitness function is the inverse of the
objective value.

\paragraph{Selection.} Although I implemented the three methods showed in class, I used \emph{montecarlo}
in the increasing phase, because this method leads to a fast convergence; instead in the diversification phase I used the linear-ranking approach, because it
mitigates the effect of the population substitution strategy, that 
saves only the best elements.

\paragraph{Combination.} For the combination I used the ordering crossover with two randomly
chosen cutting points. In order to perform this kind of crossover I use the
\verb|std::map<int,int>| data-structure defined in the header \verb|<map>|. This data-structure
has very good performance\footnote{http://en.cppreference.com/w/cpp/container/map} and most importantly it is sorted by key, i.e. the first element of the pair. 

\paragraph{Population Substitution.} During each iteration $R$ new individuals are created. Therefore
the new population is of size $N+R$.
Now the population is sorted in descending order according to the fitness value. 

Since $R$ is much smaller than $N$ and the population
at the start of the iteration is sorted, insertion sort algorithm
is used. This works very efficiently (linearly) with inputs that are almost sorted.
Instead the standard sort algorithm is (usually) quicksort that performs extremely poor with already sorted vectors $O(N+R)^2$.
Therefore only the first $N$ elements are kept, discarding
the last $R$ elements.

\paragraph{Stopping Criterion.} The algorithm terminates when it reaches 500 non improving iterations.
The drawback of this approach is that it is not predictable when
the algorithm will finish. In addition sometimes the improvements are so low that it is not worth to repeat other 500 iterations.



\subsection{Parameter calibration}
