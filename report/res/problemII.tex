\section{Second Assignment}
In this second assignment the purpose was to implement a solution for the problem
using a metaheuristic.
I chose to use a genetic algorithm. 
\paragraph{Encoding of the individuals} In order to encode the individuals I used the provided
encoding, i.e. an array of dimension N+1 containing a sequence of holes index, e.g.
$<0\ 1\ \dots\ N\ 0>$
\paragraph{Initial Population} The initial population is formed by $k$ individuals.
In order to diversify the population fast all the individuals are chosen randomly, but
the increasing sequence $<0\ 1\ \dots\ N\ 0>$ and the decreasing sequence $<0\ N\ \dots\ N\ 0>$
are always granted. In order to start with better solution some of them are trained.

\paragraph{Fitness Function} For simplicity the fitness function is the inverse of the
objective value.

\paragraph{Selection} although I implemented the three methods showed in class, I used montecarlo
in the increasing phase and linear-ranking in the diversification phase, in order to
mitigate the effect of the Population Substitution.

\paragraph{Combination} For the combination I used the ordering crossover with two randomly
chosen cutting points. In order to perform this kind of crossover I use the
\verb|std::map<int,int>| data-structure defined in the header \verb|<map>|. This data-structure
has very good performance\footnote{http://en.cppreference.com/w/cpp/container/map} and most importantly it is sorted by key, i.e. the first element. It is a RB-Tree.
With 

\paragraph{Population Substitution} During each iteration $R$ new individuals are created. Therefore
the new population is of size $N+R$, but only the $N$ best are mantained the other $R$ are discarded.
This is achieved by sorting the population by increasing value (i.e. decreasing fitness).

\paragraph{Stopping Criterion} The algorithm terminates when it reaches 500 non improving iterations, i.e. 
iterations in which there was no improvement.


\subsection{Parameter calibration}
