\section{Result}
The following results are depending also on the HW of the machine used.
The following tests were run on a  Intel(R) Core(TM) i7-3610QM CPU @ 2.30GHz.
In order to show the differences between the two implementations I selected some files
that you can find in the \verb|SAMPLE| directory.
I wrote a script \verb|benchmark.sh| that runs
each instance of the problem ten times and then it computes the
user time\footnote{In general the CPLEX API cpu time is much bigger(~7 time) than the genetic algorithm time
because CPLEX is multithreaded, the implementation of the second algorithm use only one thread}.
In order to test this problem I used the 
following instances 
\begin{itemize}
	\item TSP12, the tsp instance with 12 elements given during the class;
	\item TSP60, the tsp instance with 60 elements given during the class;
	\item REALWORLD, the real-world tsp instance, with data extrapolated from a real
	\verb|.gbr| file mentioned above;
	\item CIRCLE48, an instance generated with the InstanceGenerator that is formed by
	four very distant circles composed by 12 elements each;
	\item CIRCLE3-45 Circle instances 3 to 45;
	\item RANDOM1 a random instance, in order to show the general-purpose goodness of the algorithm;
	\item RANDOM2 a random instance, in order to show the general-purpose goodness of the algorithm;
\end{itemize}

\subsection{Result - First Assignment}




\section{Result - Second Assignment}