
\section{Introduction}
The aim of this project was to familiarize with different approaches for the solution of a well-known combinatorial problem, especially:
\begin{enumerate}
	\item An exact solution using the CPLEX API \ref{sec:problemI}
	
	\item A metaheuristic: a genetic algorithm and simulated annealing were implemented. \ref{sec:problemII}
\end{enumerate}
\subsection{Problem description}
\label{sec:problem}
The assignment of the problem was the following:
\begin{quote}
	A company produces boards with holes used to build electric frames. Boards are positioned over a 
	machines and a drill moves over the board, stops at the desired positions
	and makes the holes. Once a board is drilled, a new board is positioned and the process is
	iterated many times. Given the position of the holes on the board, the company asks us
	to determine the hole sequence that minimizes the total drilling time, taking into account
	that the time needed for making an hole is the same and constant for all the holes.
\end{quote}
This problem can be viewed as a particular instance of the TSP problem: the salesman is represented by the 
drill and the cities are represented by the holes.
Since the problem is NP-hard, the exact method does not halt in a reasonable amount of time when $N$ increases, i.e. with inputs bigger than $60$.

\paragraph{Notation} In the next sections the formalization of the assignment will be used, in particular $N$ refers to the
number of the holes and $y_{ij}$ takes the value $1$ if the arc from $i$ to $j$
is used otherwise $0$. During the rest of the report the expression \emph{the problem} will be used in order to reference.
to the assignment.

\paragraph{Structure} In the first part of the report the focus is put on instance creation, design choices of the first and second argument, particularly about the data-structures used, in the second part the attention is put on the performance and precision confrontation.
 

