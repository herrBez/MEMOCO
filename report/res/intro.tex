
\section{Introduction}
The aim of this project was to familiarize with different approach for the solution of a well-known combinatorial problem, in
particular the assignment required to solve the problem with two different approaches:
\begin{enumerate}
	\item An exact solution using the CPLEX API \url{https://www-01.ibm.com/software/commerce/optimization/cplex-optimizer/}\ref{sec:problemI}
	
	\item A metaheuristic. \ref{sec:problemII}
\end{enumerate}
\subsection{Problem description}
The assignment of the problem was the following:
\begin{quote}
	A company produces boards with holes used to build electric frames. Boards are positioned over a 
	machines and a drill moves over the board, stops at the desired positions
	and makes the holes. Once a board is drilled, a new board is positioned and the process is
	iterated many times. Given the position of the holes on the board, the company asks us
	to determine the hole sequence that minimizes the total drilling time, taking into account
	that the time needed for making an hole is the same and constant for all the holes.
\end{quote}
This problem can be viewed as a particular instance of the TSP problem: the salesman is represented by the 
drill and the cities are represented by the holes.
Since the problem is NP-hard, the exact method does not halt in a reasonable time. Therefore the metaheuristic approach is more convenient exspecially when $N$ is big, i.e. bigger than $60$.
 
\subsection{Problem Formalization}
\begin{itemize}
	\item $y_{ij}$ 1 if the edge between hole $i$ and $j$ is used.
\end{itemize}
